\documentclass{beamer}
\usepackage[utf8]{inputenc}
\usepackage[spanish]{babel}
\usepackage{hyperref}
\usepackage{verbatim}
\usepackage{listings}

\setbeamercovered{invisible}
\usetheme{Frankfurt}
\usefonttheme{serif}

% Configurar los listings (Códigos)
\renewcommand{\lstlistingname}{Código}
\lstset{
	language=C++,               % Lenguaje
	basicstyle=\ttfamily\footnotesize,  % Tipo de fuente
	keywordstyle=\color{blue},  % Color de palabras clave
	stringstyle=\color{red},    % Color de strings
	commentstyle=\color{gray},  % Color de comentarios
	showstringspaces=false,     % No muestrar el _ cuando el string tiene espacios
	breaklines = true,          % Partir las líneas largas
	breakatwhitespace=true,	    % Partir las líneas en un espacio
	numbers=left,				% Numerar las líneas a la izq
	numberstyle=\tiny,			% Poner los números de las líneas pequeños
	numberblanklines=true,      % Numerar las líneas en blanco
	columns=fullflexible,       % No perder el formato al dejar los espacios
	keepspaces=true,   			% Dejar los espacios insertados
	frame=tb,					% Poner el recuadro
}


\title{Semillero de Programación}
\subtitle{Representación de grafos, getline y stringstream}
\author{Ana Echavarría \and Juan Francisco Cardona}

\institute{Universidad EAFIT}
\date{15 de febrero de 2013}

\begin{document}

\begin{frame}
	\titlepage
\end{frame}

\begin{frame}
	\frametitle{Contenido}
	\tableofcontents
\end{frame}

\section{Representación de grafos}
	\begin{frame}
		\frametitle{Tarea}
		\begin{alertblock}{Tarea}
			Escrbir un programa que genere las representaciones como matriz y lista de adyacencia de un grafo \textbf{no dirigido} $G$.\\
			La primera línea de la entrada consiste de dos enteros $n$ el número de nodos y $m$ el número de aristas $(1 \leq n \leq 10$ y $1 \leq m \leq 100)$. Las siguientes $m$ líneas tienen cada una dos enteros $u$ y $v$ $(1 \leq u, v \leq n)$ que indican que hay una arista que une los nodos $u$ y $v$.
		\end{alertblock}
	\end{frame}
	
	\begin{frame}[fragile]
		\frametitle{Solución para la matriz}
		\begin{lstlisting}
			using namespace std;
			#include <iostream>

			const int MAXN = 15;
			int M [MAXN][MAXN];

			int main(){
			   int n, m;
			   cin >> n >> m;

			   for (int i = 0; i < m; ++i){
			      int u, v; cin >> u >> v;
			      u--; v--;
			      M[u][v] = M[v][u] = 1;
			   }
			   return 0;
			}
		\end{lstlisting}
	\end{frame}
	
	\begin{frame}[fragile]
		\frametitle{Solución para la lista de adyacencia}
		\begin{lstlisting}
			using namespace std;
			#include <iostream>
			#include <vector>

			const int MAXN = 15;
			vector <int> g [MAXN];

			int main(){
			   int n, m;
			   cin >> n >> m;

			   for (int i = 0; i < m; ++i){
			      int u, v; cin >> u >> v;
			      u--; v--;
			      g[u].push_back(v);
			      if (u != v) g[v].push_back(u);
			   }
			   return 0;
			}
		\end{lstlisting}
	\end{frame}
	
	\begin{frame}[fragile]
		\frametitle{Limpieza de los contenedores}
		\begin{block}{Limpiar los contenedores}
			Si en un problema se tienen múltiples casos de prueba, entre caso y caso hay que limpiar los contenedores porque estos están definidos como variable global.
		\end{block}
	\end{frame}
	
	\begin{frame}[fragile]
		\frametitle{Limpieza de la matriz}
		\begin{lstlisting}
			using namespace std;
			#include <iostream>

			const int MAXN = 15;
			int M[MAXN][MAXN];

			int main(){
			   int n, m;
			   while (cin >> n >> m){
			      for (int i = 0; i <= n; ++i){
			         for (int j = 0; j <= n; ++j){
			            M[i][j] = 0;
			         }
			      }   
			      // Crear el grafo
			   }
			   return 0;
			}
		\end{lstlisting}
	\end{frame}
	
	\begin{frame}[fragile]
		\frametitle{Limpieza del vector}
		\begin{lstlisting}
			using namespace std;
			#include <iostream>
			#include <vector>

			const int MAXN = 15;
			vector <int> g [MAXN];

			int main(){
			   int n, m;
			   while (cin >> n >> m){ 
			      for (int i = 0; i <= n; ++i){
			         g[i].clear(); //Limpiar la lista del nodo i
			      }
			      // Crear el grafo
			   }
			   return 0;
			}
		\end{lstlisting}
	\end{frame}
	
	\begin{frame}
		\frametitle{Matriz para grafos con pesos}
		En la representación como matriz, en lugar de almacenar un 1 en la posición $u, v$ si los nodos $u$ y $v$ están conectados, almacenar el peso con el que están conectados el nodo $u$ y el nodo $v$. Si los dos nodos no están conectados almacena $+\infty$.
		\vfill
		\begin{equation}
			M[u][v] =
			\left\{
				\begin{array}{ll}
					W_{u,v} & \mbox{si } (u, v) \in E \mbox{ o} \\
					+\infty & \mbox{en caso contrario.} 
				\end{array}
			\right.
		\end{equation}
	\end{frame}
	
	\begin{frame}[fragile]
		\frametitle{Lista de adyacencia para grafos con pesos}
		\begin{itemize}
			\item En la representación como lista de adyacencia, si el nodo $u$ está conectado con el $v$ con un peso $w$, en $g[u]$ se almacena la pareja $(v, w)$.
			\item Para esto es útil el tipo de dato \verb|pair <int, int>| que es una pareja de enteros.
			\item El grafo podría ser \verb| vector < pair <int, int> > g[MAXN]|.
			\item Para conocer más del tipo de dato \verb|pair| mirar http://www.cplusplus.com/reference/utility/pair/?kw=pair
			\item También se puede crear una estructura usando \verb|struct|.
		\end{itemize}
	\end{frame}

\section{Entrada unsando getline y stringstream}

	\begin{frame}[fragile]
		\frametitle{La función getline}
		Si la entrada es un texto y cada caso de prueba es una línea con del texto es necesario usar la función \verb|getline| para leer esa línea de la entrada. \\
		Un ejemplo de un problema donde se necesite usar \verb|getline| para leer la entrada es el problema 483 - Word Scramble del juez http://uva.onlinejudge.org
	\end{frame}
	
	\begin{frame}[fragile]
		\frametitle{Usando getline}
		\begin{lstlisting}
			using namespace std;
			#include <iostream>
			
			int main(){
			   string line;
			   while (getline(cin, line)){
			      // Do something with line
			   }
			   return 0;
			}
		\end{lstlisting}
		El programa se ejecuta mientras que se pueda leer una línea del texto de entrada. La línea se lee y se almacena en el string line.
	\end{frame}
	
	\begin{frame}[fragile]
		\frametitle{La clase stringstream}
		La clase \verb|stringstream| sirve para manipular strings como si fueran una entrada.\\
		Esta clase se utiliza cuando se tiene un string de entrada y se quiere extraer la información que tiene ese string separado por espacios.\\
		Para utilizar la clase \verb|stringstream| es necesario incluir \verb|sstream|.		
	\end{frame}
	
	\begin{frame}[fragile]
		\frametitle{Usando stringstream}
		Ejemplo: Dado un texto de varias líneas diga cuántas palabras hay en cada línea.
		\begin{lstlisting}
			using namespace std;
			#include <iostream>
			#include <sstream> // Necesario para usar stringstream
			
			int main(){
			   string line;
			   while (getline(cin, line)){
			      stringstream ss(line); // El stream ss se crea con line
			      string word;
			      int count = 0;
			      while (ss >> word) count++; // ss se usa igual que cin
			      cout << count << endl;
			   }
			   return 0;
			}
		\end{lstlisting}
	\end{frame}
	
	\begin{frame}
		\frametitle{Tarea}
		\begin{alertblock}{Tarea}
			\begin{enumerate}
				\item Registrarse en el sitio uva.onlinejudge.org
				\item Entrar al sitio http://contests.factorcomun.org/contests/49 y agregar su usuario de UVa con el botón \emph{Add new team}
				\item Resolver los 4 problemas de la competencia
				\item El alumno que quede de primero en la competencia ganará un regalo sorpresa
			\end{enumerate}
		\end{alertblock}
	\end{frame}
	
	\begin{frame}[fragile]
		\frametitle{Tarea}
		\begin{exampleblock}{Pistas}
			Algunas pistas para los problemas
			\begin{description}
				\item [Problema A] Usar el valor absoluto \verb|abs(x)|.
				\item [Problema B] Implementar el algoritmo descrito en el problema.
				\item [Problema C] Utilizar el tipo de dato \verb|double| para representar los números decimales.
				\item [Problema D] Usar \verb|getline| y luego \verb|stringstream|.
			\end{description}
		\end{exampleblock}
	\end{frame}


\end{document}